\anonsection{Введение}

Многоагентная система — это система, образованная несколькими взаимодействующими интеллектуальными агентами.
Многоагентные системы могут быть использованы для решения таких проблем, которые сложно или невозможно решить с помощью одного агента

В многоагентной системе агенты имеют несколько важных характеристик:

\begin{itemize}
    \item Автономность: агенты, хотя бы частично, независимы
    \item Ограниченность представления: ни у одного из агентов нет представления о всей системе, или система слишком сложна, чтобы знание о ней имело практическое применение для агента.
    \item Децентрализация: нет агентов, управляющих всей системой
\end{itemize}

В качестве примера многоагентной системы можно назвать игру <<Жизнь>>. 
Игра <<Жизнь>> — клеточный автомат, придуманный английским математиком Джоном Конвеем в 1970 году\cite{Gardner1970}.
Это игра без игроков, в которой человек создаёт начальное состояние, а потом лишь наблюдает за её развитием.
Игра располагается на двумерном поле, на поле располагается определенное количество одинаковых агентов.
Каждый агент живет по заданному набору правил, эти правила остаются статичны на протяжении работы клеточного автомата.

Однако игра <<играется>> на двумерном поле. Можно ли увеличить размерность поля до трехмерного и получить рабочую версию игры?
В качестве системы моделирования используется AnyLogic\cite{AnyLogic}. 


\clearpage