\section{Секция 1}

Игра <<Жизнь>> - это клеточный автомат. Каждая ячейка живет по статичному набору правил.
Место действия игры — размеченная на клетки плоскость, которая может быть безграничной, ограниченной или замкнутой. 
Каждая клетка на этой поверхности имеет восемь соседей, окружающих её, и может находиться в двух состояниях: быть «живой» (заполненной) или «мёртвой» (пустой).
Распределение живых клеток в начале игры называется первым поколением. Каждое следующее поколение рассчитывается на основе предыдущего по таким правилам:
\begin{itemize}
    \item в пустой (мёртвой) клетке, с которой соседствуют три живые клетки, зарождается жизнь;
    \item если у живой клетки есть две или три живые соседки, то эта клетка продолжает жить; в противном случае (если живых соседей меньше двух или больше трёх) клетка умирает («от одиночества» или «от перенаселённости»).
\end{itemize}

AnyLogic — это мультиметодный инструмент имитационного моделирования, разработанный компанией AnyLogic\cite{AnyLogic}.
Он поддерживает методологии моделирования на основе агентов, дискретных событий и системной динамики\cite{7822080}.

В AnyLogic в качестве примеров есть модель игры <<Жизнь>>. Она живет точно по описанным выше правилам.
У AnyLogic присутствует ограничение - он превосходно работает в двумерных плоскостях, на трехмерных плоскостях он может показывать презентации.

Для начала необходимо создать новый агент и его фигуру, назовем его <<Cell>> (см. рис. \ref{img:1}). Определим его состояния (рис. \ref{img:2}):
\begin{itemize}
    \item X,Y,Z - координаты расположения агента
    \item ID X, ID Y, ID Z - номера позиций в координатной сетке.
    \item alive - состояние ячейки: жива или нет
    \item neighbors - соседи ячейки. с течением игры не меняется.
    \item nAliveAround - количество живых соседей, с течением игры меняется
\end{itemize}

\addimghere{img1.png}{0.4}{Агенты системы}{img:1}

\addimghere{img2.png}{0.7}{Агент <<Cell>> системы}{img:2}

\addimghere{img3.png}{0.7}{Агент <<Main>> системы}{img:3}

В главной агенте <<Main>>, где будет происходить запуск модели и где они будут жить, расположим область отрисовки 3D объектов и массив с популяцией агентов (рис. \ref{img:3}).

В самом начале симуляции модели, каждой ячейке <<Cell>> в популяции назначается позиция в пространстве, ее ID и соседи (рис. \ref{img:4}).
Пространство ограничено по размеру, и границы не зациклены между собой.
Такие ограничения выбраны специально для упрощения создания модели.

У агента Cell установим значения состояния alive с помощью случайного значения с некоторым шансов.
На каждом шаге системы выполняются следующие действия (рис. \ref{img:5}):
\begin{itemize}
    \item Определяется количество выживших агентов
    \item Принимается решение о том, будет жить агент или нет
\end{itemize}

\addimghere{img4.png}{0.6}{Действия агента <<Main>> при запуске}{img:4}

\addimghere{img5.png}{0.6}{Действия агента <<Cell>>}{img:5}

Выбор правил выживания ячейки, а так же кого считать соседями - является главной составляющей модели. 
К сожалению, не удается подобрать такие правила, чтобы ячейки не погибали сразу же.

В результате получается модели многоагентной системы, в трехмерном пространстве (рис \ref{img:6}).

\addimghere{img6.png}{0.6}{Пример запуска симуляции модели}{img:6}

\clearpage

